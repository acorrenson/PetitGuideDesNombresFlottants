\documentclass{book}
\usepackage[T1]{fontenc}
\usepackage[french]{babel}
\begin{document}
  \title{Petit guide des nombres flottants}
  \author{Arthur Correnson}
  \maketitle
  \tableofcontents
  
  \chapter*{Motivations}
  \addcontentsline{toc}{chapter}{Motivations}

  Lorsque l'on souhaite écrire un programme utilisant des nombres décimaux, il est commun d'utiliser des nombres flottants pour effectuer les calculs. Il s'avère que ce recours systématique à l'usage des flottants cache des difficultés qui sont souvent méconnues ou mal comprise. En effet, l'arithmétique des nombres flottants est très différente de l'arithmétique usuelle sur les nombres réels. Il s'ensuit que les nombres flottants ne peuvent pas (et ne doivent pas!) être utilisée comme substitut aux nombres réels et leur usage doit faire l'attention d'une attention toute particulière. Par ailleurs, l'étude de l'arithmétique flottante s'avère être un sujet très riche est fondamentalement difficile. Malgré cela, force est de constater que l'étude des nombres flottants n'est enseignée ni en licence d'informatique, ni en licence de mathématique. Lorsque j'ai véritablement découvert les aspects mathématiques des nombres flottants à la fin de ma licence, quelle ne fut pas ma surprise! J'ai pris conscience que je n'avais alors aucune compréhension de ce qu'est un nombre flottant et que la plupart des programmes flottants que j'avais pu écrire était tout à fait incorrectes.

  En écrivant ce livre, j'espère transmettre du mieux que possible quelques éléments clefs pour mieux comprendre les nombre flottants et leur utilisation. N'étant pas un expert mais un simple curieux, je n'écris pas ce livre pour qu'il soit perçu comme un cours mais plutôt comme un recueil de remarques et d'explications que j'aurais souhaité qu'on me transmette plus tôt dans mes études d'informatique.

  Cette ouvrage s'adresse à tout le monde. L'arithmétique flottante est un sujet que je trouve personnellement difficile, et les ressources disponibles à son sujet nécessitent un certain bagage mathématique pour pouvoir être appréciée. A l'inverse, je souhaite qu'une partie non négligeable de cette ouvrage soit intelligible même sans prérequis en informatique ou en mathématique.
  Toutefois, pour satisfaire la curiosité de tout un chacun, je souhaite également fournir des détails techniques, des exemples d'algorithmes et des preuves lorsque cela est possible et aide à la compréhension.

  Enfin, il me tient à coeur que ce ce livre soit complètement open-source. Les contributions sont les bienvenues et j'espère qu'elle seront nombreuses.

  \chapter{Introduction}


  \chapter{Introduction}
\end{document}